\RequiredExercise{Noch mehr Multimedia}
%
\par Für diese Aufgabe können Sie zum Beispiel den Trailer zu Sintel\footnote{\small{©} copyright Blender Foundation  \textbar \vspace{1em} \url{www.sintel.org}} verwenden. Sie finden das Video unter\\
\url{https://download.blender.org/durian/trailer/sintel_trailer-480p.mp4}
\\(MP4-Format) bzw. \\
\url{https://download.blender.org/durian/trailer/sintel_trailer-480p.ogv} \\
(OGG-Vorbis). Verwenden Sie auch in dieser Aufgabe beide Quellen!
%
\par Ändern Sie nun Ihre Webseite aus Aufgabe 17 so, dass anstelle eine
Soundausgabe eine Videoausgabe platziert wird. Sie müssen im Prinzip nur den
\htag{audio}-Tag durch den \htag{video}-Tag ersetzen. Fügen Sie nun allerdings
die Standardkontrollelemente über das Attribut controls hinzu. Des Weiteren
sollen Sie ein (beliebiges) Vorschaubild über das \jvar{poster}-Attribut
angeben.
%
\par Entfernen Sie die Buttons für Stopp und Abspielen / Pause. Diese sind nun
über die Standardkontrollelemente eingebaut. Die Buttons für die
Abspielgeschwindigkeit dagegen sollen weiterhin verwendbar sein. Fügen Sie zum
Abschluss noch Behandlung für einige Ereignisse hinzu: Beim Auslösen des
\jvar{ontimeupdate} Ereignisses sollen Sie die aktuelle Videozeit in einem
\htag{div} auf der Seite anzeigen. Beim Auslösen des \jvar{onended} Ereignisses
soll eine Meldung, z.B.
%
\begin{lstlisting}
alert('Video abgespielt.');
\end{lstlisting}
%
ausgegeben werden.