\RequiredExercise{Komplexe Zahlen in JavaScript}
%
\par Sie sollen eine Klasse zur Rechnung mit komplexen Zahlen erstellen. Nennen
Sie diese Klasse \jvar{cmplx}. Komplexe Zahlen bestehen aus dem sog. Realteil
und dem sog. Imaginärteil. Die Trennung erfolgt durch die Konstante $i$, welche
durch $i^2 \equiv -1$ definiert ist. Wenn $a$, $b$, $c$ und $d$ reelle Zahlen
sind (d.h. Datentyp \emph{Number} in JavaScript), so ergibt sich für
Multiplikation und Addition, dass
%
\begin{eqnarray}
(a + b i) + (c + d i)     & = & (a + c) + (b + d) i,\\
(a + b i) \cdot (c + d i) & = & (ac - bd) + (ad + bc) i.
\end{eqnarray}
%
\par Ihre Klasse sollte daher folgende Eigenschaften besitzen:
%
\begin{itemize}
\item
Realteil (genannt \jvar{re})
\item
Imaginärteil (genannt \jvar{im})
\end{itemize}
%
\par Fügen Sie folgende Methoden hinzu:
%
\begin{itemize}
\item
\jfunc{add} zum Addieren von Zahlen mit der aktuellen Instanz, wobei über die
Variable \jvar{arguments.length} die Anzahl der übergebenen Argumente überprüft
werden soll. Bei zwei Argumenten sollen die Zahlen als Real- und Imaginärteil
interpretiert werden. Bei einem Argument soll das Argument als komplexe Zahl
interpretiert werden. Führen Sie die Addition wie oben gezeigt aus.
%
\item
\jfunc{multiply} zum Multiplizieren von Zahlen mit der aktuellen Instanz. Gehen
Sie hierbei analog zur Implementierung von \jfunc{add} vor, wobei Sie anstelle
der Addition die Multiplikation wie oben gezeigt implementieren.
\end{itemize}
%
\par Der Konstruktor der Klasse soll bis zu zwei Elemente (Real- und
Imaginärteil) erlauben. Denken Sie daran, dass in JavaScript Argumente optional
sind und geben Sie den Argumenten daher als Standardwert $0$.
%
\par Testen Sie Ihre Klasse über einige Rechnungen in der JavaScript Konsole
Ihres Webbrowsers.