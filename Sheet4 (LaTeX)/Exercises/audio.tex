\RequiredExercise{Multimedia mit \htag{audio}}
%
\par Erstellen Sie eine Webseite die ein \htag{audio}-Element, sowie vier Buttons (Stopp, Abspielen / Pause, Langsamer, Schneller) enthält. Für das Element sollen folgende Angaben gelten:
%
\begin{itemize}
\item Erstellen Sie das Element ohne Controls.
\item Geben Sie die Quelle nicht über das \jvar{src}-Attribut an, sondern über \htag{source}-Elemente.
\item Folgende Quellen sollen verwendet werden: \url{http://html5.florian-rappl.de/sample.mp3} (MP3-Format) und \url{http://html5.florian-rappl.de/sample.ogg} (OGG Vorbis-Codec).
\item Das Element soll über die Buttons durch JavaScript gesteuert werden.
Um den letzten Punkt zu verwirklichen müssen Sie das \jvar{onclick}-Ereignis der Buttons setzen. Um die entsprechenden Funktionalitäten zu implementieren werden Sie folgende Methoden des Audioelements benötigen:
\item Die Eigenschaften paused und ended gibt Ihnen Auskunft über den aktuellen Zustand.
\item Mit den Methoden \jfunc{play} und \jfunc{pause} steuern Sie den aktuellen Zustand. Die aktuelle Zeit setzen Sie über die Eigenschaft \jvar{currentTime}\footnote{In einigen Browser ist das Setzen von \jvar{currentTime} und \jvar{playbackRate} noch nicht implementiert.}.
\item Mit der Eigenschaft playbackRate können Sie die Geschwindigkeit einstellen ($1$ entspricht hier \qty{100}{\%}, d.h. normale Geschwindigkeit).
\end{itemize}